\documentclass{ximera}
%% You can put user macros here
%% However, you cannot make new environments

\listfiles

\graphicspath{{./}{firstExample/}{secondExample/}}

\usepackage{tikz}
\usepackage{tkz-euclide}
\usepackage{tikz-3dplot}
\usepackage{tikz-cd}
\usetikzlibrary{shapes.geometric}
\usetikzlibrary{arrows}
\usetkzobj{all}
\pgfplotsset{compat=1.13} % prevents compile error.

%\renewcommand{\vec}[1]{\mathbf{#1}}
\renewcommand{\vec}{\mathbf}
\newcommand{\RR}{\mathbb{R}}
\newcommand{\dfn}{\textit}
\newcommand{\dotp}{\cdot}
\newcommand{\id}{\text{id}}
\newcommand\norm[1]{\left\lVert#1\right\rVert}
 
\newtheorem{general}{Generalization}
\newtheorem{initprob}{Exploration Problem}

\tikzstyle geometryDiagrams=[ultra thick,color=blue!50!black]

%\DefineVerbatimEnvironment{octave}{Verbatim}{numbers=left,frame=lines,label=Octave,labelposition=topline}



\usepackage{mathtools}

\author{Anna Davis \and Dino Seketa} \license{CC-BY 4.0}

\title{Where was Eye?} \license{CC BY-NC-SA 4.0}

\begin{document}
\begin{abstract}
We determine the location of the camera based on a photograph it took.
\end{abstract}
\maketitle

\section*{Where was Eye?}

\subsubsection*{Abstract}
In this activity we will determine the height of the camera based on the photograph it took. 

\subsubsection*{Target Audience}
\begin{itemize}
    \item Teacher development workshops (grades 6 - 12)
    \item Geometry students (grades 6 - 12)
    \item Math Circles/extracurricular activities
\end{itemize}

Participants need to be able to (1) solve ratio equations, e.g. $\frac{a}{b}=\frac{x}{c}$; (2) discuss similar triangles in terms of ratios.

\subsubsection*{Standards}
Similar triangles, measurement and units, ratios and proportions, real-life applications, modeling. 

\subsubsection*{Materials and Facilities}
Facilities requirements:
\begin{itemize}
    \item Instructor computer access (with internet access, PowerPoint or OneNote).
    \item Access to long tables/desks or a long hallway.
\end{itemize}

Required materials: 
\begin{itemize}
    \item One ruler for each group of 2 - 3 students.
    \item One smart phone for each group of 2 - 3 students.
    \item Pencils and paper. 
\end{itemize}

Optional materials:
\begin{itemize}
    \item One yard stick/meter stick for each group of 2 - 3 students.
    \item One 8-inch by 10-inch piece of plexiglass for each group of 2 - 3 students.
    \item One fine-tip dry-erase marker for each group of 2 - 3 students.
\end{itemize}

\subsubsection*{Description of Activity}
This activity consists of three parts.  In Part 1, students discover a relationship between some geometric properties of a given photo and the height at which the photo was taken.  Students work with a given photo, make computations related to similar triangles, and establish a method for estimating the height of the camera.  Students can use the online or printed worksheet to complete this part.  

In Part 2, students replicate the experiment in Part 1 by taking their own photos, and performing measurements on the photos.  Students will need phone cameras and rulers for this part.

In Part 3, students will discover the principles of visual perspective which explain why the formula in Part 1 works.

\subsubsection*{Instructor Notes}

This activity can be adapted to fit a variety of settings.  The following are suggestions for three scenarios.

\begin{itemize}
    \item \textbf{One 45-minute lesson.} Parts 1 and 3 can be completed without engaging in the hands-on portion of the activity.  
    \item \textbf{Two 45-minute lessons.} All three parts can be completed.
    \item \textbf{Math Circle Student Activity.}  All parts can be completed, but the emphasis should be placed on Part 2.  For this reason, Exploration \ref{exp:hallway} can be skipped, and Exploration \ref{exp:trianglesAtBase} should be done in the context of students' own photos.  Group discussion questions from Exploration \ref{exp:hallway} can still be used.
\end{itemize}


\end{document}