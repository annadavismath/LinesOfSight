\documentclass{ximera}
\input{../preamble.tex}

\title{Where was Eye? Part I} \license{CC BY-NC-SA 4.0}

\begin{document}

\begin{abstract}
\end{abstract}
\maketitle

\section*{Where was Eye? Part I}

\begin{exploration}\label{exp:matchPic}
Three students, Adam, Benjamin, and Cayla, took part in a photography competition.  All three submitted photos of railroad tracks.  Adam and Benjamin took photos from their natural height; Cayla flew a drone high above her head to take a picture.
    \begin{image}
         \includegraphics[width=5in]{AdamBenjaminCayla.jpg}
\end{image}
The images Adam, Benjamin, and Cayla took appear below (in no particular order).
\begin{image}
         \includegraphics[width=5in]{threePics.jpg}
\end{image}
Match each image with the photographer who took it.

Photo 1 was taken by 
\begin{multipleChoice}
    \choice{Adam}
    \choice{Benjamin}
    \choice[correct]{Cayla}
\end{multipleChoice}

Photo 2 was taken by 
\begin{multipleChoice}
    \choice[correct]{Adam}
    \choice{Benjamin}
    \choice{Cayla}
\end{multipleChoice}

Photo 3 was taken by 
\begin{multipleChoice}
    \choice{Adam}
    \choice[correct]{Benjamin}
    \choice{Cayla}
\end{multipleChoice}

In small groups, discuss the reasons for your choices.  Do you think it is possible to use these photographs to estimate the height of the camera that took each photo?
\end{exploration}

\begin{exploration}\label{exp:trianglesAtBase}
    Let's look at the three pictures geometrically.  The rails appear to meet at a single point.  
    
\begin{image}
         \includegraphics[width=5in]{vanishingPoint.jpg}
\end{image}
    
    Do you know what this point is called?
    \begin{multipleChoice}
        \choice{Center point}
        \choice[correct]{Vanishing point}
        \choice{Vertex}
        \choice{End point}
    \end{multipleChoice}

The rails, together with the bottom of each photo form triangles.  

\begin{image}
         \includegraphics[width=5in]{triangles.jpg}
\end{image}

You might describe the first triangle as tall and narrow, and the second triangle as wide and short.  These descriptions refer to the \emph{proportions} of these triangles.  To really see the difference in the proportions, we can make the bases of the triangles the same size.  This makes sense to do because railroad tracks have the same width everywhere in the U.S.

\begin{image}
         \includegraphics[width=3.5in]{matchedTriangles.jpg}
\end{image}

In groups, discuss the relationship between the height of the camera and the height of the triangles.

\begin{hint}
    Note that Cayla's triangle (triangle 1) is the tallest, while Adam's triangle (triangle 2) is the shortest.  Recall that Cayla used a drone to take her photo from high above her head.  
\end{hint}

\end{exploration}

\begin{exploration}\label{exp:hallway}
In groups of two or three, use a phone to take photos of a long desk, a sheet of paper or a hallway holding your phone at different heights, as shown below.

\begin{image}
         \includegraphics[width=5in]{threeHallways.jpg}
\end{image}

To keep track of the camera height, we recommend that you hold or tape the phone to a meter stick or a ruler. Have one group member record the distance to the camera lens for every shot.  

Import your shots into PowerPoint or OneNote.  Use the ruler tool (under ''Draw") to outline the edges of the hallway in each photo. Form a triangle with the vanishing point as the top vertex, as shown below.

\begin{image}
         \includegraphics[width=2in]{ruler.jpg}
\end{image}

\begin{image}
         \includegraphics[width=5in]{hallsAndTriangles.jpg}
\end{image}

Observe that the triangles we formed are nearly isosceles.  This will be important for our future computations.  If your triangles are not isosceles, take a few more pictures, making sure that your camera lens is in the middle of the hallway (desk).

\begin{question}
    Compare the proportions of the three triangles above. Which camera was in the highest position?  Which was in the lowest position?  The vanishing point is conveniently located on the door.  Compare the locations of the vanishing point across the three photos.  What happens to the vanishing point as the height of the camera decreases?

    \begin{multipleChoice}
        \choice[correct]{As the camera lowers, the vanishing point also lowers.}
        \choice{As the camera lowers, the vanishing point rises.}
        \choice{As the camera lowers, the vanishing point does not change location.}
    \end{multipleChoice}
\end{question}


\begin{image}
         \includegraphics[width=3in]{threeTriangles.jpg}
\end{image}

\begin{image}
         \includegraphics[width=4.5in]{triangleMeasures.jpg}
\end{image}



\end{exploration}

\end{document}