\documentclass{ximera}
\input{../preamble.tex}

\title{Where was Eye? (part 3)} \license{CC BY-NC-SA 4.0}

\begin{document}

\begin{abstract}
We determine the location of the camera based on a photograph it took.
\end{abstract}
\maketitle

\section*{Where was Eye? (part 3)}


We will now develope a theoretical foundation for our method of figuring out the height of the camera. 

\begin{exploration}\label{exp:hideAndSeek}
Suppose Alice, Bob, Colin, Daria, and Evan are playing hide-and-seek in the school yard.  Alice is the seeker.  The diagram below shows the location of the players.  Which of the children is visible to Alice?
     \begin{image}
         \includegraphics[width=5in]{hideAndSeek1.jpg}
\end{image}
Check the name of all children that Alice can see.
\begin{selectAll}
\choice[correct]{Bob}
\choice[correct]{Colin}
\choice{Daria}
\choice{Evan}
\end{selectAll}

\textbf{Group Discussion Prompt:}
\emph{Articulate the reason for your choices.  If you need help formulating your thoughts, click on the arrow (below, right), and use the diagram to help you.}

\begin{expandable}
    \begin{image}
         \includegraphics[width=4in]{hideAndSeek2.jpg}
\end{image}
\end{expandable}
\end{exploration}

The above exploration intuitively established a very important fact: we see along straight lines.  These lines are called \emph{lines of sight}.
\begin{image}
         \includegraphics[width=4in]{linesOfSight.jpg}
\end{image}

\subsection*{Picture Planes}
To understand how lines of sight can help us create a realistic picture, imagine a canvas made of glass.  If you position the glass canvas in front of the object you want to draw, you can simply trace the object onto the glass with a marker.  We will call the glass canvas the \emph{picture plane}.

Take a look at the photograph below.  The student in the photo has just finished tracing the cube onto the glass.  From her point of view, the tracing matches up with the cube. 

\begin{image}
         \includegraphics[width=3in]{cube.jpg}
\end{image}

Now let’s draw lines of sight that connect the corners of the cube with the eye. The line of sight from each corner of the cube passes through its image on the glass!  

\begin{image}
         \includegraphics[width=3in]{cubeLines.jpg}
\end{image}

This principle will allow us to draw any object from the point of view of an imaginary eye.

If we were to place a camera where the student's eye was located, the picture taken by the camera would match the tracing on the glass.

\subsection*{Vanishing Point and the Height of the Eye (Camera)}




\end{document}